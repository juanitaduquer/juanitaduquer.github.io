\documentclass[12pt]{article}

%% Language and font encodings
\usepackage[english]{babel}
\usepackage[utf8x]{inputenc}
\usepackage[T1]{fontenc}

\usepackage{pgf,tikz,pgfplots}
\pgfplotsset{compat=1.13}
\usepackage{mathrsfs}
\usetikzlibrary{arrows}
\usepackage{mathtools}


%% Sets page size and margins
\usepackage[letterpaper,top=1in,bottom=1in,left=1in,right=1in]{geometry}
\setlength{\parindent}{0cm}

%% Useful packages
\usepackage[autostyle]{csquotes}
\usepackage{amsmath,amssymb,dsfont,mathrsfs}
\usepackage{amsthm}
\usepackage{tcolorbox}
\usepackage{tikz-cd}

\usepackage{enumitem}

\usepackage[all,cmtip]{xy}


\newenvironment{alphabetize}{\begin{enumerate}\def\theenumi{\alph{enumi}}}{\end{enumerate}}

% \usepackage{graphicx}
\usepackage{pgfplots}
\pgfplotsset{compat=newest}

% \usepackage[colorlinks=true, allcolors=blue]{hyperref}
% \newcommand{\C}{{\mathbb{C}}}
\newcommand{\N}{{\mathbb{N}}}
\newcommand{\Q}{{\mathbb{Q}}}
% \newcommand{\A}{{\mathbb{A}}}
% \newcommand{\R}{{\mathbb{R}}}
% \newcommand{\Ell}{{\mathscr{L}}}
\newcommand{\Z}{{\mathbb{Z}}}
% \newcommand{\K}{\mathbb{K}}
% \newcommand{\F}{\mathds{F}}
% \newcommand{\RP}{\ensuremath{\mathbb{R}\mathrm{P}}}
% \newcommand{\CP}{\ensuremath{\mathbb{C}\mathrm{P}}}
% \newcommand{\jac}{\ensuremath{\mathrm{Jac}}}
% \renewcommand{\div}{\ensuremath{\mathrm{div}}}
% \newcommand{\ord}{\ensuremath{\mathrm{ord}}}
% \newcommand{\tr}{\ensuremath{\mathrm{Tr}}}
% \newcommand{\gal}{\ensuremath{\mathrm{Gal}}}
% \newcommand{\Oo}{\ensuremath{\mathcal{O}}}
% \newcommand{\cl}{\ensuremath{\mathrm{CL}}}
% \newcommand{\Hom}{\ensuremath{\mathrm{Hom}}}
% \newcommand{\End}{\ensuremath{\mathrm{End}}}
% \newcommand{\idn}{\mathfrak{n}}
% \newcommand{\idm}{\mathfrak{m}}
% \newcommand{\idp}{\mathfrak{p}}
% \newcommand{\idr}{\mathfrak{r}}
% \newcommand{\idq}{\mathfrak{q}}
% \newcommand{\tensor}{\otimes}
% \newcommand{\norm}[1]{\left\|#1\right\|}
% \newcommand{\abs}[1]{{\left|#1\right|}}
% \def\inv{^{-1}}
% \def\iso{\cong}
% \newcommand{\til}[1]{{\widetilde{#1}}}
% \DeclareMathOperator{\spec}{Spec}
% \DeclareMathOperator{\mor}{Mor}

% \newcommand{\Wedge}{\bigwedge}
% \newcommand{\dual}{^\vee}



% \renewcommand{\O}{\ensuremath\mathcal{O}}
% \renewcommand{\P}{\ensuremath{\mathds{P}}}
% %% Contents table
\usepackage[colorlinks]{hyperref}

\usepackage{fancyhdr}
\usepackage{color}
\definecolor{qqzzcc}{rgb}{0.,0.6,0.8}


\hypersetup{
  colorlinks   = true,
  linkcolor = qqzzcc,
  citecolor    = magenta
}


% %% Theorems
% % \theoremstyle{plain}
% \newtheorem{theorem}{Theorem}[section]
% % \newtheorem{proposition}[theorem]{Proposition}
% \newtheorem{lemma}[theorem]{Lemma}
% % \newtheorem{corollary}[theorem]{Corollary}

% \theoremstyle{definition}
% \newtheorem{definition}[theorem]{Definition}
% \newtheorem{example}[theorem]{Example}

% % \theoremstyle{remark}
% % \newtheorem*{remark}{Remark}

% \usepackage{bm}
\usepackage{colonequals}
\usepackage{verbatim}

\begin{document}

\begin{center}
\begin{Large}
    \textbf{Random \textsf{Magma} knowledge}\\
\end{Large}
\end{center}

This is a compilation of some useful \textsf{Magma} tricks that I have learned through the years. Thank you to all the wonderful people who have shared their knowledge with me:
\href{https://math.dartmouth.edu/~eassaf/}{Eran Assaf},
\href{https://math.mit.edu/~edgarc/}{Edgar Costa},
\href{https://sachihashimoto.github.io}{Sachi Hashimoto},
\href{https://math.dartmouth.edu/~akulkarn/}{Avi Kulkarni},
\href{https://math.mit.edu/~sschiavo/}{Sam Schiavone},
\href{https://sites.google.com/view/pim-spelier/home}{Pim Spelier},
\href{https://math.mit.edu/~drew/}{Drew Sutherland}, and
\href{https://math.dartmouth.edu/~jvoight}{John Voight}.

\begin{itemize}
\item Useful links: \href{http://magma.maths.usyd.edu.au/magma/handbook/}{Documentation} and \href{http://magma.maths.usyd.edu.au/magma/pdf/examples.pdf}{general examples}.

\item You can write the first letter of a function and tab complete twice to get a list of possible functions.
\item To kill a process, use  \verb|control| + \verb|c|. To kill \textsf{Magma}, do this twice.
\item To ignore \verb|>|, use \verb|SetIgnorePrompt(true);|. This is very helpful when you are copying code from the terminal.
\item \verb|$1| denotes the last printed result (you can also call \verb|$2| and \verb|$3|).
\item The rational numbers are not a number field in \textsf{Magma}. You can instead call

\verb|RationalsAsNumberField();|.  Careful: linear algebra is slower here.

\item When you are debugging functions, you can \verb|SetDebugOnError(true);|. This will give you access to the "inside" of your function, up to where \textsf{Magma} got stuck. You print things by \verb|p whateverYouWant|.  Use \verb|q| to quit back to the \textsf{Magma} terminal.  You can see all the functions that were used in your computation using \verb|bt| (shows you the "frames").  Go to a specific frame by writing \verb|f theNumberYouWant|.  Warning: do not use \verb|;|.

\item In your home directory you can create a \verb|file.Magmarc|, if does not exist, to have certain commands to run on start.  To find the file, you can go to your home directory and press Command + Shift + . (period).


\item Use \verb|control| + \verb|e| to get to the last character of a line in the terminal. Use \verb|control| + \verb|a| for the first one. Do \verb|control| + \verb|k| to delete the line.  You can also find other combinations \href{http://magma.maths.usyd.edu.au/magma/handbook/text/53}{here}.

\item Do \verb|%p| to print all the Magma session.
\item You can search only for signatures without inheritance:

\verb|ListSignatures(ModFrmHilElt : Isa := false);|.  Moreover, you can also just look for functions where your type is an argument or a return values (very useful when you try to find a function producing the type that another function needs…):

\verb|ListSignatures(ModFrmHilElt : Search := "ReturnValues", Isa := false);|.

\item \textsf{Magma} complains about types.  Elements need to be coerced to have the type you want. For example, to coerce and integer or type Rational  to type Integer: \verb|Integers()!r|.

\item The function \verb|Discriminant(K)| returns the discriminant of the polynomial used to define, not the discriminant of $K$. Use \verb|Discriminant(RingOfIntegers(K))| to get the discriminant of $K$.

\item Use for verbose: \verb|vprintf VerboseLevel : "whatever you want to print";|

\item To iterate over a list, you can use \verb|& + (operation)|.  For example, to add all the prime numbers $\le 20$, do: \begin{verbatim}&+[p : p in [1..20] | IsPrime(p)];\end{verbatim}

\item To run a magma Jupyter kernel, go to \href{https://github.com/edgarcosta/magma_kernel}{this GitHub repository}, and then you can run \verb|sage -n jupyter| .

\item To print in a format that is readable by \textsf{Magma}: \verb|Sprint(whateverYouWant, "Magma");|.

\item To coerce a list into the same set: 
\verb+[Integers() | 4/1, 5/1]+ .

\item To load a \textsf{Magma} file line by line do \verb|iload "NameOfFile.m";|.
\end{itemize}
Do you have more random knowledge? Please email it to me and I will add it to this list!
\end{document}
